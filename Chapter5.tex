%!TEX root = ../thesis.tex
%*******************************************************************************
%****************************** Fifth Chapter *********************************
%*******************************************************************************




\chapter{Conclusion and future work} % 540 words

Initial work has focused on developing procedures for azurite and verditer sample preparation and analysis by Raman spectroscopy, SEM-EDS, and AFM-IR. The analytic procedures are non-destructive and do not damage samples. Although sample preparation by embedding and polishing is not reversible, only very small amounts of sample material is required for analysis. An initial survey of thirteen loose pigment samples and one historical easel painting cross section established the utility of Raman spectroscopy and SEM-EDS for the identification of mineral components of the samples and the characterisation of their morphology. Particle morphology, size distribution, and trace element analysis are parameters whose use is supported in the discrimination between natural and synthetic pigment samples and mineral sample provenancing. These have shown to be able to be characterised by the combination of analytic methods used here. These preliminary conclusions represent a basis upon which further analysis of historic wall painting samples will build.

Additional samples of natural azurite within geologic formations as well as historic easel painting cross sections containing both natural azurite and synthetic blue verditer pigments will be analysed by the same methods to develop a more complete library of statistically relevant variation in samples depending on their natural or synthetic origins. Significant time has been dedicated to securing these samples, and I must acknowledge the generosity of the Hamilton Kerr Institute and the Sedgwick Museum of Earth Sciences for offering samples for analysis. In the future, wall painting samples will also be supplied in order to address the central research question of the origin of early modern blue pigments used in vernacular wall paintings. 

Initial work also includes the results of a collaborative project with Katherine Waldron at the Hamilton Kerr Institute, addressing the identity and morphology of blue pigments appearing in \textit{Battle of Spurs}, a Tudor-era painting. Twelve cross sections removed from different areas of the work were analysed using SEM-EDS, allowing identification of pigments as well as likely associate minerals. Azurite and smalt were identified as the blue pigments used. The dimensions of azurite particles in each cross section were measured and a statistical analysis of particle size and shape (skew) was carried out. This project is ongoing, and initial results show that it is possible to determine variation in particle size and shape as well as chemical composition in different paint layers and canvas locations. An analysis of the pigment behaviour in binder depending on particle size and shape would be an interesting route to pursue in the future. 

Potential interesting questions about the use of copper carbonates/azurite as pigments remain and may be addressed further into the project. These include attaining a better understanding of the optical effects of pigment size and shape as well as binder interactions which could also affect the reflective index of the paint layer containing azurite particles. These also include a discussion of provenancing of natural azurite, as this work has been limited in the past due to sample sizes available. This will build on existing literature concerning pigment mineral provenancing that has primarily focussed on lapis lazuli (ultramarine) and cinnabar (vermilion). Provenancing work to date has relied upon trace element analysis. SEM-EDS as well as ToF-SIMS would therefore likely be used, depending on sample requirements. 