%!TEX root = ../thesis.tex
%*******************************************************************************
%****************************** FOURTH Chapter **********************************
%*******************************************************************************
\chapter{Conclusion and further work}

% **************************** Define Graphics Path **************************
\ifpdf
    \graphicspath{{Chapter4/Figs/Raster/}{Chapter4/Figs/PDF/}{Chapter4/Figs/}}
\else
    \graphicspath{{Chapter4/Figs/Vector/}{Chapter4/Figs/}}
\fi

% **************************** Chapter text **************************

\section{Conclusion}
\label{section4.1}

This work consists of two strands of research. First, the capability of using spatially-offset Raman spectroscopy to detect and analyse low concentration lampblack carbon glazes was investigated. Glaze samples were constructed using historically significant materials and procedures and analysed using a defocused confocal Raman system to simulate true SORS. Poor detection of pigments due to very strong absorption of the incident beam and surface damage at laser powers sufficient to generate a Raman signal has led us to conclude that this system, limited by strong absorption in dark pigments, is not suitable for this type of analysis. Work was also done to confirm earlier spatially-offset results on two-layer paint samples. The low SORS effect observed on systems containing CdS/TiO\textsubscript{2} shows that this system is capable of analysing some pigment systems, but is not suitable for analysis of lampblack pigment. While there is interesting further work to be done developing more sensitive SORS systems, we believe this will require development of true spatially-offset detection and that this system may be broadly unsuitable for analysis of carbon-based black pigments.

The need to speed sample preparation for use on the SORS system brought up interesting questions regarding sample ageing and the effects of changing various ageing conditions on the resulting linseed oil chemistry. The oxidation of linseed oil, known to depend on environmental conditions and oil additives, is well studied. However, significant gaps in the literature remain regarding the effects of certain pigments on oxidation rates and surface structure. Lampblack carbon pigment has been minimally studied. After the negative results from SORS experiments, we pivoted to an investigation of the effects of lampblack pigment and ultraviolet light exposure on oxidation of a mixture of linseed oil and mastic resin. This work made use of transmission IR spectroscopy as well as AFM-IR and SFG spectroscopy, novel surface analytic methods that have not been widely applied in conservation science.

Results of this work shows that addition of lampblack pigment does affect the oxidation rate of mastic varnish, resolving the uncertainty raised by existing competing sources. The addition of lampblack pigment slows oxidation by approximately one week in thin films. Exposure to ultraviolet light speeds oxidation significantly but also leads to further reactions not observed under natural conditions within the time period available for this experiment. Interesting domain structures were observed in all samples using AFM, and were affected by the presence of lampblack as well as ultraviolet light exposure, with significant intercalation of domain boundaries into domain centers in samples that were exposed to ultraviolet light. SFG spectroscopy suggests that surface conformational order is affected by the addition of lampblack pigment, a very interesting result that merits further study.

\section{Further work}
\label{section4.2}

This work will be continued to resolve unanswered questions raised by AFM and SFG results. While many studies have addressed the oxidation of linseed oil using bulk analytic techniques such as transmission IR, we propose using the novel surface specific methods of SFG and AFM-IR to determine the chemical significance of the surface domains observed in this work. 

In particular, spatially-resolved SFG spectroscopy would determine whether local conformational order is affect by proximity to lampblack particles and how conformational order changes as films oxidise and form a hardened crosslinked structure. Although we have recently struggled to collect useful spectra using AFM-IR, this issue will be resolved in the future and the resolution this method offers will help characterise the changes in surface chemistry accompanying oxidation and identify the cause of the formation of surface domains that seem to vary depending on oxidation and sample conditions.

The identification of specific oxidation products, particularly of mastic resin, is also of interest. Time of flight Secondary Ion Mass Spectrometry (ToF-SIMS) offers some spatial resolution that would allow the selection of samples for mass spectrum analysis from specific sample regions, and this would allow determination of whether different components are segregating on the surface into domains and oxidising \textit{via} different pathways.  

Finally, further work will also address the question of throughdry, or the depth of penetration of oxidation. Throughdry has posed significant problems for artists and industry, since the formation of an oxidised surface layer under certain conditions prevents diffusion of oxygen deeper into the paint film. Determination of the penetration depth of the ultraviolet light into the sample as well as variable angle attenuated total reflectance (ATR) spectroscopy would be useful in determining the degree and rate of throughdry in different samples. 




