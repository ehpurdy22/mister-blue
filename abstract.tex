% ************************** Thesis Abstract *****************************
% Use `abstract' as an option in the document class to print only the titlepage and the abstract.
\begin{abstract} % 332 words
Azurite, a naturally occurring copper carbonate, was an important blue pigment used since antiquity. As it was costly to procure, the development of alternative blues was an important issue for artists and patrons. The expansion of the use of blue pigments in England during the early modern period, in particular in the context of vernacular wall paintings, suggests the development and scaling of production of blue verditer, azurite's synthetic analogue. Limited research has been done into the spread of these pigments in wall paintings and, more generally, into methods that can distinguish a natural or synthetic origin. 

This report presents a study of reference azurite and verditer pigments to establish baseline expectations about the morphology and elemental composition of natural and synthetic samples as well as sample preparation and analysis procedures. Scanning electron microscopy and energy dispersive and Raman spectroscopy are used. The purpose of this initial work is to lay a foundation, and further research will use these results to analyze unknown samples collected from wall paintings in the East of England to establish the identity of pigments and their spread throughout the region. 

This report also presents scientific analysis of paint cross sections removed from \textit{Battle of Spurs}, a Tudor-era painting containing numerous areas of azurite pigment. This work sought to understand the pigment preparation procedures used across the painting, with a particular focus on particle size and shape as well as associate minerals appearing alongside azurite in the work. 

Further work will broaden our understanding of the characteristics of natural azurite mineral samples as well as synthetic verditers used in easel paintings. Wall painting samples will be studied using this framework, and results from SEM-EDS and Raman spectroscopy will be cross-correlated to draw out possible connections between morphological characteristics and chemical composition. Particle behaviour and optical characteristics in binders may also be studied. Finally, questions remain about the possibility of provenancing azurite samples based associate mineral content and trace element analysis, and future work on this may be possible.

\end{abstract}
